\section*{Resumé}
I denne rapport, og dette projekt generelt, bliver en RISC-V implementering af Rainbow signatur systemet fremlagt. Rainbow er en finalist i NIST standardiseringsprocessen for \emph{post-quantum} kryptografiske algoritmer i kategorien for digitale signatur systemer. Den overførte version af reference implementeringen vil blive gennemgået og testet for antallet af CPU cykler og instruktioner. Grundlaget for projektet var derfra at opnå kendskab til \emph{post-quantum} kryptologi, med særligt fokus på Rainbow, samt at give potentielle optimeringer til systemet med CPU cykler og instruktioner i fokus.
\medskip\\
De optimeringer der blev forsøgt implementeret til Rainbow inkluderede \emph{bitslicing} og \emph{opslagstabeller}. Ingen af disse to optimeringer formåede at opnå et lavere antal cykler end standard (reference) implementeringen (efter at være oversat til RISC-V platformen). Dog opnåede \emph{bitslicing} systemet en bedre cykel/instruktion-ratio end de to andre versioner (\emph{opslagstabeller} og reference) til trods for et markant større antal cykler. Antallet af cykler i den pågældende version for bitslicing nåede 3300\% flere cykler end standard implementering. Til trods for dette står denne "optimering" som et \emph{fundament} for at afprøve yderligere optimering, særligt på \texttt{C}-kode i et højere niveau, med relation til bitslicing.

\pagebreak