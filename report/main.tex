% Information and metadata
% --------------------------

	\newcommand{\ccourse}{BADM500}							% \ccouse - course code (and name)
    \newcommand*\xor{\oplus}

% Preamble
% --------------------------
	% BEGIN_FOLD

	% Document Setup
	% ------------------------
	\documentclass[11pt,a4paper]{article}       		% Optional = twoside for two-sided documents

	% Packages
	% ------------------------
		% BEGIN_FOLD

			% Essential Packages
			\usepackage[english]{babel}					% Language pack
			\usepackage[utf8]{inputenc}					% Input encoding for UTF8

			% Mathematics Packages
			\usepackage{amsmath}						% Basic math tools
			\usepackage{amssymb}						% Basic math symbols

			% Programming Packages
			\usepackage{listings}						% Listings provide code environments
			\usepackage{courier}						% Font package for code

			% Layout Packages
			\usepackage{lastpage}
			\usepackage{graphicx}						% Importing .svg, .pdf and other
			\usepackage{caption}						% Custom figure captions
			\usepackage{fancyhdr}						% Package for fancy headers and footers
			\usepackage{float}							% Float options for placing images, tables, etc.
			\usepackage{courier}						% Courier font for code
			\usepackage{subcaption}						% Separate captions for subimages
			\usepackage{color}                          % Gives colour commands (Used for links)
			\usepackage[pdftex, colorlinks]{hyperref}   % Makes hyperlinks work and coloured
			\usepackage[paperwidth=8.5in]{geometry}     % Enlarges the width of the page by inches
			\usepackage[noabbrev,capitalize]{cleveref}

			% Drawing and diagram packages
			\usepackage{tikz}							% Package for drawing diagrams

			% Misc. Packages
            \usepackage{nameref}                        % Package for referencing names
            \usepackage{leading}                        % Package for giving the lines spacing
            \usepackage{comment}                        % Allows to comment entire paragraph
            \usepackage{tabularx}
            \usepackage[ruled,vlined]{algorithm2e}
            

		% END_FOLD

	% Shortcuts and commands
	% ------------------------
		% BEGIN_FOLD
  			\newcommand{\code}[1]{\lstinline|#1|}							% \code{} Insert code here to have in-line code.
			
			\newcommand{\br}{\newline\newline}                              % Adds an empty line


		% END_FOLD

	% Settings / Preferences
	% ------------------------
		% BEGIN_FOLD

			% Equation, align, and gather environments can have page breaks
			\allowdisplaybreaks

			% Coloured links (Optional, delete for non-coloured links)
			\hypersetup{colorlinks,
				citecolor = black,
				filecolor = black,
				linkcolor = black,
				urlcolor = blue
				}

			% Colours for code environment
			\definecolor{codegreen}{rgb}{0,0.6,0}
			\definecolor{codegray}{rgb}{0.5,0.5,0.5}
			\definecolor{codepurple}{rgb}{0.58,0,0.82}
			\definecolor{backcolour}{rgb}{0.95,0.95,0.92}

			% Configure code environment
			\lstset{
				numbers			= left,
				stepnumber 		= 5,					% Numbers every 5 lines of code
				firstnumber 	= 1,
				numberfirstline = true,
				showtabs 		= false,
				basicstyle		= \ttfamily,
				breaklines		= true,
				columns			= fixed,
				showspaces		= false,
				commentstyle	= \color{codegreen},
				keywordstyle	= \color{magenta},
				numberstyle		= \color{codegray},
				stringstyle		= \color{codepurple},
				showstringspaces= false
				}

			% Nicer dots for itemisation
			\renewcommand{\labelitemi}{·}				% Nicer dots for 'itemize' environments

            
		% END_FOLD

	% END_FOLD

% Layout (Header and footer can be edited here)
% --------------------------
	% BEGIN_FOLD

		% Margins
		\addtolength\textheight{2cm}
		\addtolength\topmargin{-1cm}
		\addtolength\marginparwidth{1.5cm}
		\addtolength\headheight{1.6pt}

		% Indent and paragraphs
		\setlength\parindent{0pt} 								% Comment to indent first line in  paragraph
		
		% Line spacing
		\leading{17pt}

		% Header, footer
		\pagestyle{fancy}										% Template for header and footer
		%\fancyhf{}												% Set header and footer to empty

		% Left header (inner if using twopage layout)
		\fancyhf[HLO]{University of Southern Denmark}

		% Center header
		\fancyhf[HC]{\ccourse}

		% Right header (outer)
		\fancyhf[HRO]{01-06-2021}

		% Left footer (inner)
		%\fancyhf[FLO,FRE]{}

		% Center footer
		\fancyhf[FC]{Page \thepage \, of \pageref{LastPage}}

		% Right footer (outer)
		%\fancyhf[FRO,FLE]{}

	% END_FOLD
	
% ==========================
% Document
% ==========================
\begin{document}

% 	\maketitle 													% Prints title, author and date
% 	\begin{minipage}[t]{0.6\textwidth}
%         \begin{flushleft} \large
%             \emph{Author:}\\
%             \text{Mikkel Juul Vestergaard}
%         \end{flushleft}
%     \end{minipage}
%     \begin{minipage}[t]{0.4\textwidth}
%         \begin{flushright} \large
%             \emph{Supervisor:} \\
%             \text{Assist. professor,}\\
%             \text{Ruben Niederhagen}
%         \end{flushright}
%     \end{minipage}\\[3cm]

    \begin{titlepage}
    \begin{center}
        \vspace*{1cm}
            
        \LARGE
        \textbf{Post Quantum Cryptography on the RISC-V Platform - Rainbow}
            
        \vspace{2cm}
        
        \Large
        \emph{Author:}\\
        Mikkel Juul Vestergaard
        
        \vspace{0.5cm}
        \emph{Supervisor:}\\
        Assist. Professor, Ruben Niederhagen
        \vfill
        
        \vspace{0.8cm}
            
        \includegraphics[width=0.6\textwidth]{resources/SDU.jpeg}
            
        \Large
        Department of Mathematics and Computer Science\\
        University of Southern Denmark\\
        Denmark\\
        01-06-2021
            
    \end{center}
\end{titlepage}


    %\includegraphics[width]{resources/SDU.png}
    \thispagestyle{empty}% No header or footer on first page
    \pagenumbering{gobble} 
   	%\thispagestyle{fancy}										% First page with header and footer
    \newpage
    \thispagestyle{empty}
    \tableofcontents %Indholdsfortegnelse

    \newpage
    \pagenumbering{arabic} 
    \nocite{*}
    % Include new tex files here.
    % --------------------------
        \section*{Resumé}
I denne rapport, og dette projekt generelt, bliver en RISC-V implementering af Rainbow signatur systemet fremlagt. Rainbow er en finalist i NIST standardiseringsprocessen for \emph{post-quantum} kryptografiske algoritmer i kategorien for digitale signatur systemer. Den overførte version af reference implementeringen vil blive gennemgået og testet for antallet af CPU cykler og instruktioner. Grundlaget for projektet var derfra at opnå kendskab til \emph{post-quantum} kryptologi, med særligt fokus på Rainbow, samt at give potentielle optimeringer til systemet med CPU cykler og instruktioner i fokus.
\medskip\\
De optimeringer der blev forsøgt implementeret til Rainbow inkluderede \emph{bitslicing} og \emph{opslagstabeller}. Ingen af disse to optimeringer formåede at opnå et lavere antal cykler end standard (reference) implementeringen (efter at være oversat til RISC-V platformen). Dog opnåede \emph{bitslicing} systemet en bedre cykel/instruktion-ratio end de to andre versioner (\emph{opslagstabeller} og reference) til trods for et markant større antal cykler. Antallet af cykler i den pågældende version for bitslicing nåede 3300\% flere cykler end standard implementering. Til trods for dette står denne "optimering" som et \emph{fundament} for at afprøve yderligere optimering, særligt på \texttt{C}-kode i et højere niveau, med relation til bitslicing.

\pagebreak
        \section{Introduction}
The modern world of computing has grown to become a largely impactful aspect of everyones lives. Due to this increasing impact it is not uncommon for perpetrators to attack the digital structures that societies rely on. Whether it is against personal computing or more societally important computing functions, some seek to abuse the importance of modern day computing for financial, activist, or other types of gain.\medskip\\
As the rise of the public internet, connecting billions of people and organizations, made this even more prominent and easy for perpetrators to abuse, cryptologists, mathematicians and computer scientists all over have tried to minimize the risk of using a public internet of computing for important tasks like banking, business confidentials, personal health, etc.\medskip\\
The main aspects of this digital cryptology are public-key cryptography, symmetric-key cryptography and hashing. These three aspects each play their different roles in securing the lives and information of billions of people globally. Depending on the information and computing power available, these will be used for different purposes, as they each provide their own unique take on private life on the public internet.\medskip\\
For securing sensitive information from external readers, symmetric key algorithms or public-key algorithms are typically used. A symmetric-key encryption scheme is fast and typically has small key-sizes, though it relies on both parties knowing the key in advance. A public-key scheme allows communicating parties to exchange public keys and encrypting data to eachother through these, typically sacrificing speed and key-sizes.\medskip\\
For storing passwords and other personal information it is typical to use hashing as a one-way "encryption" of data, such that only the user itself knows what this information is. Hashing can also be used to verify the authors of certain publicly shared documents, by incorporating public-key schemes into the mix and therefore enabling anyone to use a mixture of a public-key encryption and hashing to digitally sign a document.\medskip\\
As researchers world-wide constantly look for new ways of computing, like DNA computing, photonic computing, quantum computing, and more, the cryptographic strength of standardized in-use cryptographic algorithms has also been challenged. The strengths of some modern computing-types like quantum computing over standard electrical computers have shown their ways into the cryptographic scene.\medskip\\
Algorithms like RSA and Diffie-Helmann key-exchange, standard and elliptic-curve, are both well-known cryptographic algorithms that will become obsolete once sufficiently strengthed and stable quantum computers have been built. Currently many big tech-companies are working on quantum computers, and the strength of each is ever increasing.\medskip\\
A quantum computer computes quite differently from a standard computer, as it uses \emph{qubits} instead of the more well-known \emph{bits}. As many might be aware of, a standard computer works by having a central processing unit and some storage local to it, some persistent and some not. The central processing unit works, on a virtual scale, by using values of 0s and 1s to determine its actions and as the values that these actions act upon.\medskip\\
A quantum computer, and the qubits that it works with, does not consider either a 0 or a 1 value whenever it reads some qubits. The qubits can be implemented using electrons or photons. The idea of quantum computing is then to harness aspects of quantum physics, such as using superposition to have qubits represent multiple combinations of 0s and 1s at the same time, whereas a typical computer would have a string of bits be a single combination. Due to such aspects as superposition, entanglement and decoherence, sufficiently-sized quantum computers have been proven to provide a significant speedup in some problems over standard electrical computers, some of which relate to the security of modern day cryptographic algorithms.\medskip\\
Although such computers might seem like they would have to replace standard electrical computers, the current state of cryptography is to build cryptographic schemes that are secure against quantum computers, but can still be ran on a typical personal computer without significant disadvantage. These algorithms, also called \emph{post-quantum cryptographic algorithms}, have been under a standardization process by NIST for a few years, by 2021.\medskip\\
As quantum computers are not here to replace every typical electric computer in use, the global technology market has to adapt products to use these \emph{post-quantum} algorithms. One such aspect of the technology market is embedded computing. Embedded devices are gradually more and more normal to see in everyday-homes and therefore must also follow this security trend. Embedded devices can be found in modern vehicles, home appliances, robot- and drone controllers, and more, whilst general low power-consumption hardware can be found in mobile phones, tablets, laptop computers and alike in general. All aforementioned require small low power-consuming processing units for providing their core functionalities.\medskip\\
Though, the problem with embedded devices is that they typically have to run low power-consumption hardware with small computational footprints. As it is inevitable that most post-quantum cryptographic algorithms are not made directly for embedded hardware it can be quite a lot heavier to run post-quantum tasks on embedded hardware, meaning that more localized optimizations for embedded hardware implementations is in need.\medskip\\
This project seeks to bridge the gap between embedded devices and the more computationally heavy post-quantum cryptographic algorithms. In particular, the project takes a look at a scheme used for digital signatures, using multivariate polynomials as part of the underlying mathematics.
\pagebreak
        \section{Preliminaries}
\subsection{The Rainbow scheme}
Rainbow is a multivariate public key cryptosystem (MKPC) and can be seen as a multi layered version of the Unbalanced Oil and Vinegar scheme [NIST rainbow round 3 paper]. Rainbow is a potential candidate in round three of the NIST post-quantum standardization for signature schemes. Many MKPCs make use of the NP-hardness of the MQ problem to provide security. [NIST rainbow round 3 paper]
\medskip\\
The keys in Rainbow consists of systems of different kinds of maps. The private key holds a system, $F$, of quadratic polynomials which is then composed with two affine invertible maps, $S$ and $T$, that are supposed to hide the structure of $F$. The public key, $P$, of the system is then $P = S \circ F \circ T$. The multivariate polynomials of Rainbow are defined over different finite fields, according to which security level is chosen. In this project, security level \textit{I} is used and the polynomials are therefore defined over the values of $GF(2^4)$.[NIST rainbow round 3 paper, Multivariate public key cryptosystem chapter 2]
\medskip\\
Signature generation in rainbow is done by using the maps inverted maps of the private key and computing in a \textit{chain-like} way where the next result relies on the previous. For verifying a signature one simply has to compute the hash of the document (of course, using the same hashing function as the signer), compute the output of $P(\textbf{z})$, the public key with the signature as input, and check if the two values are equal. [NIST rainbow round 3 paper]

\subsection{The RISC-V ISA}
RISC (Reduced Instruction Set Computer) architectures are typically optimized with regard to register use (either through the compiler or through actual hardware registers), instruction pipelines, and having small and fewer instructions (compared to CISC machines).\medskip\\
One such RISC architecture is the RISC-V architecture. The RISC-V architecture is an open standard and as such can be freely implemented for different kinds of usage. Any RISC-V implementation is supposed to be some combination of the base RISC-V ISA, which is limited but still usable for many general purposes, and optional extensions providing features like atomic instructions, multiplication and division, floating point support and more. A subset of the extensions and their base ISAs can be seen in [RISC-V Offers Simple, Modular ISA] or [Design of the RISC-V Instruction Set Architecture].
\medskip\\
For this project, the processor used for benchmarking is a VexRiscV-based platform called PQVexRiscV simulated on a host machine running a Ryzen 7 5800X. The CPU is openly available and was originally created for, the RISC-V counterpart to the pqm4 library, pqriscv [PQVexRiscV reference]. The VexRiscV CPU in general is an RV32IM instruction set CPU (32-bit integer base ISA with the multiplication and division extension) running a five stage pipeline (fetch, decode, execute, memory, writeback). The VexRiscV CPU is optimized for FPGA use [VexRiscV docs] and as such is interesting with regard to embedded or industrial use (for now at least). Other than usage of the PQVexRiscV CPU, this project is not affiliated with the pqriscv project nor the VexRiscV project.
        \section{Implementation}
The following section details the differences that came about when porting the reference implementation[Link to github of rainbow] to a platform without an operating system, minimal standard library, and more constrained on memory. It will also detail some of the specifics that the original authors used to implement the functionality of Rainbow for reference in later sections. It will, however, not go into detail on how the optimizations tested differ from the original code, see section \ref{opti} for this.
\subsection{Overview of the reference system}
In general, this port of the Rainbow reference implementation tries to stay as close to its roots as possible, making it easier to argue for correctness and easier to port these findings to other platforms as well (having multiple sources being somewhat similar).\medskip\\
The implementation focuses solely on the standard Rainbow implementation with no key-size adjustments. Focusing on this particular variant of the Rainbow scheme makes the optimizations done clearer for readers new to the Rainbow scheme in general, as it is the closest to how Rainbow is described in many sources on Rainbow and the $MQ$ problem and therefore should be somewhat easier than also having to understand some of the key-size reductions going on in \texttt{CZ-Rainbow} and \texttt{Compressed-Rainbow}.

\subsubsection{Utilities}
Due to the constrained nature of an embedded device and/or FPGA, not all the original functionality provided by the Rainbow authors was needed. In particular, dynamic memory allocation was removed entirely due to the somewhat unstable state of using standard library dynamic memory allocation on embedded devices. Instead, all memory allocations is handled as static memory allocation. Alternatively, one could have implemented a sufficient and secure \texttt{malloc} implementation, though this was not deemed necessary for this project.\medskip\\
The original Rainbow reference implementation also relied on having all data stored in files. As the device running this port of Rainbow has only \texttt{512 kb} of ROM and RAM (each), this was changed. This port of Rainbow relies on a host machine connecting to the embedded device and providing signatures, keys, messages and seeds for (cryptographically secure) randomness.\medskip\\
Although the above two aspects has been altered, they were not aspects that the Rainbow scheme internally was heavily reliant on. 
\subsection{Deep dive into reference functionality}
        \section{Optimizations} \label{opti}
For this project multiple approaches were tested for running Rainbow on a constrained-computing platform such as those typically seen in embedded devices. Section \ref{opt:gf16comp} goes into detail on how elements of GF16 are treated for the optimizations of this project. It also specifies the notation used in \cref{opt:lookup} and \cref{opt:bitslice}. The latter two sections each show how the theoretical aspect of the optimization works as well as showing how it was implemented.
\subsection{GF(16) Computation} \label{opt:gf16comp}
For the subsequent sections after this the notation assumes a bit of knowledge on how a $GF(16)$ element can be treated. Any element in $GF(16)$ can be represented as a \emph{polynomial of polynomials}. What is meant by this is that any element in $GF(16)$ can take the form
$$
    \alpha a + \beta
$$
where $\alpha$ and $\beta$ both are of the form 
$$
    \gamma b + \delta
$$
such that the \emph{outermost} polynomial is linear over variable $a$ and the two \emph{innermost} are linear over the variable $b$. This is the same as expressing a $GF(16)$ element as a first-degree polynomial with first degree polynomials as coefficients.\medskip\\
One way of approaching such a representation for $GF(16)$ is then to have the coefficients of the \emph{outermost} polynomial be $GF(4)$ elements, which themselves are represented as the first degree polynomials over the variable $b$. The coefficients of the \emph{innermost} polynomial would then be $GF(2)$ elements, holding exactly a binary value.\medskip\\
By the above constructions it is then clear that a $GF(16)$ value can be represented using two, layered, polynomials where the innermost has binary coefficients. An example of such a $GF(16)$ coefficient could be
$$
    (1b + 0) a + (1b + 1) = (1b)a + (1b + 1)
$$
Further, for the following subsections I expand upon this by writing such a polynomial either as the bitstring (using the above element as reference)
$$
    1011
$$
or using square brackets to denote the coefficients of the innermost polynomial
$$
    [10]a + [11]
$$
\subsection{Lookup tables} \label{opt:lookup}
For a system over $GF(16)$ coefficients, the elements might take one of 16 different values. Throughout the publicmap computation, multiplication over such $GF(16)$ values are done quite a few times. Though, most modern architectures, and especially the \texttt{RISC-V} architecture, have not been built to support arithmetic over such finite fields. For this reason, to test whether or not the bit-manipulation approach in the reference code is a bottleneck, I used a basic lookuptable approach to compute the multiplication result of two values in $GF(16)$.\medskip\\
Computing the $16 \times 16$ table of $GF(16)$ elements was done using the original multiplication code and running it for each combination of $GF(16)$ values that exist. 
The table looks as \cref{lookuptable} and lies in the \texttt{gf16.h} file, in the function \texttt{gf16\_mul()}.
\begin{figure}[h]
    \centering
    \begin{tabular}{|c|c|c|c|c|c|c|}
        \hline
            \textbf{Polynomials} & \textbf{0000} & \textbf{0001} & \textbf{0010} & \textbf{0011} & \dots & \textbf{1111} \\
        \hline
            \textbf{0000} & 0000 & 0000 & 0000 & 0000 & \dots & 0000\\
        \hline
            \textbf{0001} & 0000 & 0001 & 0010 & 0011 & \dots & 1111\\
        \hline
            \vdots & \vdots & \vdots & \vdots & \vdots & $\ddots$ & \vdots\\
        \hline
            \textbf{1111} & 0000 & 1111 & 0101 & 1010 & \dots & 1001\\
        \hline
    \end{tabular}
    \caption{A brief view at some of the lookuptable elements}
    \label{lookuptable}
\end{figure}\\
Since each $GF(16)$ can be stored in one \texttt{uint8\_t} each, it is possible to have at most 256 (plus, potentially some overhead bytes) bytes of space used for a table like above. Given a constrained system like this, a space-speed tradeoff have to be considered thoroughly. This tradeoff will be evaluated further in \cref{evalsec}.\medskip\\
The specific implementation uses two $GF(16)$ elements, $a$ and $b$, as input and indexes the table using $a*16 + b$, as both $a$ and $b$ are promised to maximally use four bits of memory. This, of course, needs some external checks on the amount of bits used in $a$ and $b$ to be fully secure, though it was not seen as a major insecurity on the system to be addressed immediately.\medskip\\
Should it be a problem using \texttt{256} bytes of memory, some memory can be saved by not computing multiplications where one polynomial is of value $0$ but just returning a $0$ byte to the caller. The tradeoff here is then that the indexing of the table might become a bit more intricate than it is now plus an additional check is done before any actual lookup is done.
\subsection{Bitslicing} \label{opt:bitslice}
\subsubsection{Design of a Bitsliced System for GF(16)} \label{bitslice:theory}
\subsubsection{Implementation of the Bitsliced Scheme} \label{bitslice:implementation}
        \section{Evaluation} \label{evalsec}
Optimizing code to use special techniques does not ensure that the final code will run faster than the original code. For this reason, the following section will look into what is gained and what is lost by each optimization type. The primary evaluation will be on CPU cycles, as the timed benchmarks done might not be fully correct, due to overhead between the communicating parties (especially the possible overhead of the extra computation done by the test environment). Asking for cycles used on the \texttt{PQ-VexRiscV} CPU is easily done using the included \texttt{hal} setup. \textbf{how much can we talk about the hal?}
\subsection{Performance of the Reference Implementation}
\subsection{Performance of the Lookup Table Implementation}
\subsection{Performance of the Bitsliced Implementation}
        \section{Conclusion}
The goal of this project was to look into the reference implementation of the Rainbow scheme, port it to RISC-V and see if some aspects could be optimized. As the reference implementation offers multiple variants of the Rainbow scheme, each with multiple security levels, the standard variant of Rainbow level I was chosen. Trying to optimize the other two variants of Rainbow is not impossible, though would start to push the required memory sizes of the embedded device, which is quite an important aspect to take into account when working with embedded devices.
\medskip\\
The embedded platform, the PQVexRiscV CPU, used in this project is a less powerful type of embedded device when compared to some of the more mainstream ARM processors. The platform is also not as well-supported in terms of relevant libraries (for this project specifically, it might very well be for others). In total the platform is constrained to a degree for which the expectations were not to obtain a tremendously fast optimization, though to obtain some results that can be developed upon for further use. For this reason as well, the project itself is situated in an open (or at least going to be open) github repository available for further developments.
\medskip\\
As was shown in \cref{evalsec}, the speeds of the optimizations were to a degree where \emph{optimization} might be a far-fetched statement. The bitsliced implementation constructed for this project were about 3300\% slower than the standard implementation when ported. On the other hand, the lookup table implementation made for this project obtains only around 1.17\% slowdown when compared to the standard, ported, reference. The latter of the two optimization schemes were also the one closest to the reference implementation in terms of computational scheme. The lookup table implementation only replaces $GF(16)$ multiplications, whereas the bitsliced implementation replaces the whole public map computation of the scheme using a, possibly, naive way of computing it. In \cref{implementation:ffa}, referring to \cite{rainbownist}, it was stated that the computational scheme for computing the public map originally was the scheme due to Tung Chou which seeks to minimize the multiplications of finite field elements, which typically is also the more expensive of addition and multiplication.
\medskip\\
As the bitsliced optimization scheme is platform-independent and very abstracted, any further developments on trying to improve the scheme itself should not be hard as long as they follow the general structure of the code. If it is found that the scheme itself is very poorly thought out, then a total revision would be possible, potentially including some RISC-V assembly and more \texttt{c}-specific optimizations. As could be seen in \cref{bitslice:theory} the idea of bitslicing itself on a theoretical plan should be able to at least obtain the same speed as the ported reference implementation. Taking into account that the Rainbow verification scheme heavily uses finite field arithmetic, for good reason, the emphasis on the programmer-induced error in the current bitslicing scheme is heavily enforced. An interesting sidenote would be the overall cycle per instruction ratio that, if not a coincidence, could yield interesting results on the speed of the overall scheme.
\medskip\\
However, due to a limited scope, with regard to both time and complexity, any conclusive answers on the \emph{correct} nature of a bitslicing approach for this RISC-V platform cannot be provided as much time went into understanding, porting and trying to optimize the reference code. For the same reason, any further developments on expanding the research that went into optimization through lookup tables were also rather limited. One thing that this project clearly did bring was a ported RISC-V reference implementation that stands very close to the original non-ported version.
\pagebreak
        \phantomsection
        \addcontentsline{toc}{section}{References}
        \bibliographystyle{plain}
        \bibliography{bibliography}
        
\end{document}

