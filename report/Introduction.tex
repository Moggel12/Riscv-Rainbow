\section{Introduction}
The modern world of computing has grown to become a largely impactful aspect of everyones lives. Due to this increasing impact it is not uncommon for perpetrators to attack the digital structures that societies rely on. Whether it is against personal computing or more societally important computing functions, some seek to abuse the importance of modern day computing for financial, activist, or other types of gain.\medskip\\
As the rise of the public internet, connecting billions of people and organizations, made this even more prominent and easy for perpetrators to abuse, cryptologists, mathematicians and computer scientists all over have tried to minimize the risk of using a public internet of computing for important tasks like banking, business confidentials, personal health, etc.\medskip\\
The main aspects of this digital cryptology are public-key cryptography, symmetric-key cryptography and hashing. These three aspects each play their different roles in securing the lives and information of billions of people globally. Depending on the information and computing power available, these will be used for different purposes, as they each provide their own unique take on private life on the public internet.\medskip\\
For securing sensitive information from external readers, symmetric key algorithms or public-key algorithms are typically used. A symmetric-key encryption scheme is fast and typically has small key-sizes, though it relies on both parties knowing the key in advance. A public-key scheme allows communicating parties to exchange public keys and encrypting data to eachother through these, typically sacrificing speed and key-sizes.\medskip\\
For storing passwords and other personal information it is typical to use hashing as a one-way "encryption" of data, such that only the user itself knows what this information is. Hashing can also be used to verify the authors of certain publicly shared documents, by incorporating public-key schemes into the mix and therefore enabling anyone to use a mixture of a public-key encryption and hashing to digitally sign a document.\medskip\\
As researchers world-wide constantly look for new ways of computing, like DNA computing, photonic computing, quantum computing, and more, the cryptographic strength of standardized in-use cryptographic algorithms has also been challenged. The strengths of some modern computing-types like quantum computing over standard electrical computers have shown their ways into the cryptographic scene.\medskip\\
Algorithms like RSA and Diffie-Helmann key-exchange, standard and elliptic-curve, are both well-known cryptographic algorithms that will become obsolete once sufficiently strengthed and stable quantum computers have been built. Currently many big tech-companies are working on quantum computers, and the strength of each is ever increasing.\medskip\\
A quantum computer computes quite differently from a standard computer, as it uses \emph{qubits} instead of the more well-known \emph{bits}. As many might be aware of, a standard computer works by having a central processing unit and some storage local to it, some persistent and some not. The central processing unit works, on a virtual scale, by using values of 0s and 1s to determine its actions and as the values that these actions act upon.\medskip\\
A quantum computer, and the qubits that it works with, does not consider either a 0 or a 1 value whenever it reads some qubits. The qubits can be implemented using electrons or photons. The idea of quantum computing is then to harness aspects of quantum physics, such as using superposition to have qubits represent multiple combinations of 0s and 1s at the same time, whereas a typical computer would have a string of bits be a single combination. Due to such aspects as superposition, entanglement and decoherence, sufficiently-sized quantum computers have been proven to provide a significant speedup in some problems over standard electrical computers, some of which relate to the security of modern day cryptographic algorithms.\medskip\\
Although such computers might seem like they would have to replace standard electrical computers, the current state of cryptography is to build cryptographic schemes that are secure against quantum computers, but can still be ran on a typical personal computer without significant disadvantage. These algorithms, also called \emph{post-quantum cryptographic algorithms}, have been under a standardization process by NIST for a few years, by 2021.\medskip\\
As quantum computers are not here to replace every typical electric computer in use, the global technology market has to adapt products to use these \emph{post-quantum} algorithms. One such aspect of the technology market is embedded computing. Embedded devices are gradually more and more normal to see in everyday-homes and therefore must also follow this security trend. Embedded devices can be found in modern vehicles, home appliances, robot- and drone controllers, and more, whilst general low power-consumption hardware can be found in mobile phones, tablets, laptop computers and alike in general. All aforementioned require small low power-consuming processing units for providing their core functionalities.\medskip\\
Though, the problem with embedded devices is that they typically have to run low power-consumption hardware with small computational footprints. As it is inevitable that most post-quantum cryptographic algorithms are not made directly for embedded hardware it can be quite a lot heavier to run post-quantum tasks on embedded hardware, meaning that more localized optimizations for embedded hardware implementations is in need.\medskip\\
This project seeks to bridge the gap between embedded devices and the more computationally heavy post-quantum cryptographic algorithms. In particular, the project takes a look at a scheme used for digital signatures, using multivariate polynomials as part of the underlying mathematics.
\pagebreak