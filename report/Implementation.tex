\section{Implementation}
The following section details the differences that came about when porting the reference implementation[Link to github of rainbow] to a platform without an operating system, minimal standard library, and more constrained on memory. It will also detail some of the specifics that the original authors used to implement the functionality of Rainbow for reference in later sections. It will, however, not go into detail on how the optimizations tested differ from the original code, see section \ref{opti} for this.\medskip\\
The files and functionality gone through in this section are what is left from porting the implementation to RISC-V. If any changes were made to a file or functionality it will be described in some subsection except, still, for any optimizations.
\subsection{Overview of the reference system}
In general, this port of the Rainbow reference implementation tries to stay as close to its roots as possible, making it easier to argue for correctness and easier to port these findings to other platforms as well (having multiple sources being somewhat similar).
\medskip\\
The variant worked on for this project is solely the standard Rainbow variant with no key-size adjustments and a security level of I. Focusing on this particular variant of the Rainbow scheme makes the optimizations done clearer for readers new to the Rainbow scheme in general, as it is the closest to how Rainbow is described in many sources on Rainbow and the $MQ$ problem and therefore should be somewhat easier than also having to understand some of the key-size reductions going on in \texttt{CZ-Rainbow} and \texttt{Compressed-Rainbow}
\subsubsection{File overview}
The files of the reference implementation can be divided into different parts according to what aspect of the Rainbow scheme they partake in. In total, we give an overview of the functionality of this Rainbow port by dividing into these parts and explaining each part separately. First we give an overview of the files in \texttt{src/Ia\_Classic\_Reference}.
\medskip\\
One set of files of this port that had to have quite a lot of changes made to it was the files prepended with \texttt{utils}. These file are to a varying degree part of the internal structure of Rainbow. These files handle random number generation, hashing and $host\rightarrow client$ communication. The hashing aspect of this set is mostly just an API-wrapper for another hashing utility located in \texttt{src/libcrypto}. The changes made to this set of files, and the reasonings, can be seen in section \ref{utils}.
\medskip\\
Another set of files that can be sectioned as a unit of participation is the set consisting of \texttt{rng.c}, \texttt{rng.h}, \texttt{api.h} and \texttt{sign.c}. These files provide NIST-compliance as the NIST standardization requires participating schemes to provide certain functions as an API. [NIST PQC API Notes] These files have not been changed and the output of this project should therefore still be compliant with the NIST standardization API requirements. See subsection \ref{deepdive} for a walkthrough if this API.
\medskip\\
Remaining are two sets of files in \texttt{src/Ia\_Classic\_Reference}, the \texttt{rainbow} files and the library files. The \texttt{rainbow} files consist of all files where \texttt{rainbow} is in the name. These files construct the actual inner-workings of Rainbow and therefore quite well represent how Rainbow works. That is, they make use of all the other files in \texttt{src/Ia\_Classic\_Reference} to provide a simple API to the outermost \emph{tool}-files (more on those later).
\medskip\\
The last set of files in \texttt{src/Ia\_Classic\_Reference} is, as already stated, the library files. These remaining files contain code to run the, mostly, mathematical aspects of the Rainbow scheme. As can be seen in section \ref{deepdive}, Rainbow relies on linear algebra and finite field arithmetic. The linear algebra implementations like Guassian elimination, matrix-vector products, vector-vector products, etc. are implemented with finite field arithmetic and therefore all rely on some of the functionality given in \texttt{gf16.h}.
\medskip\\
Moving one directory back, to \texttt{src/}, it becomes clear that most of the Rainbow scheme is implemented in \texttt{src/Ia\_Classic\_Reference}. The only two sets of files of interest are the files prepended with \texttt{rainbow-} and the files in \texttt{src/libcrypto}. The files in \texttt{src/libcrypto} are \texttt{archive} and \texttt{header} files, of an AES and SHA256 implementation, meant for use in some of the files in \texttt{src/Ia\_Classic\_Reference}. The reasoning behind having these files is given in section \ref{utils}.\medskip\\
At last are the files \texttt{src/rainbow-*}. These files provide the three different top-level functionalities of rainbow, namely generating keypairs, signing and verifying messages. Should this be a more cohesive construction of rainbow, then one might make a tool that combines these three files into one providing an executable that can seamlessly switch between the functions on-chip instead of having to flash the ROM each time. Though, this might force the ROM usage to be larger than the \texttt{512kB} already specified.
\subsubsection{Utilities} \label{utils}
Due to the constrained nature of an embedded device and/or FPGA, not all the original functionality provided by the Rainbow authors was needed. In particular, dynamic memory allocation was removed entirely due to the somewhat unstable state of using standard library dynamic memory allocation on embedded devices. Instead, all memory allocations are handled as static memory allocation. Alternatively, one could have implemented a sufficient and secure \texttt{malloc} implementation, though this was not deemed necessary for this project.\medskip\\
The original Rainbow reference implementation also relied on having all data stored in files. As the device running this port of Rainbow has only \texttt{512kB} of ROM and RAM (each), this was changed. This port of Rainbow relies on a host machine connecting to the embedded device and providing signatures, keys, messages and seeds for (cryptographically secure) randomness.\medskip\\
Although the above two aspects have been altered, they were not aspects that the Rainbow scheme internally was heavily reliant on. As the original implementation relied heavily on hashing and AES functionality from the OpenSSL library tweaks had to be made for this port. The OpenSSL library is quite large, extensive and has no port for the RISC platform that this project is working on.
\medskip\\
The aspects of OpenSSL used originally by the Rainbow authors were the AES and SHA implementations. The rainbow reference implementation uses either of three SHA-2 variants according to the Rainbow security level. [Rainbow round 3 paper] As this project only focuses on Level I Rainbow, a replacement was needed for the SHA-256 algorithm. The replacement for both the SHA256 implementation and the AES implementation was ones that are used in the official RISC-V github repositories for testing the RISC-V scalar cryptographic instruction set extension proposals currently being looked at. [Link to github repository] These had reference implementations where no tricks or special RISC-V instructions were needed, and as such were good candidates for benchmarking on a \textit{reference} system.

\subsection{Deep dive into reference functionality} \label{deepdive}
For this subsection we will be taking a look shortly what the NIST API enforces, how key generation is actually implemented, how exactly it is that this implementation evaluates the public map on some input and how it signs a document using the inverted maps.
\subsubsection{NIST API}
For public-key signatures, NIST specifies an API consisting of two files and some functionality regarding randomness. The first file specified by NIST is the \texttt{api.h} file. This file lies in the \texttt{src/Ia\_Classic\_Reference} folder and specifies the size of the secret- and public-key, the algorithm name and the byte-overhead allowed for the signed message (how many additional bytes might it take).
\medskip\\
The second file is the \texttt{sign.c} file consisting of \texttt{crypto\_sign()}, \texttt{crypto\_sign\_open()} and \texttt{crypto\_sign\_keypair()} each of which runs a certain aspect of the Rainbow scheme. Respectively, this file consist of the signing mechanism, the verification mechanism and the key-generation mechanism. The functions are also exactly following the NIST-specified declarations.
\medskip\\
The last file is also a pair of header and \texttt{c}-file. These are the \texttt{rng.c} and \texttt{rng.h} files. These files consist of definitions and declarations for \texttt{randombytes()}, \texttt{randombytes\_init()}, \texttt{seedexpander()}, \texttt{seedexpander\_init()} and \texttt{struct AES\_XOF\_struct}. The only thing reworked for this was the AES implementation used for the seed expansion function, else this was neither touched.
\subsubsection{Key generation}
An overview of the pseudo-code behind the key generation scheme can be seen in [NIST rainbow round 3 paper]. 
\subsubsection{Message signing}

\subsubsection{Signature verification}
\subsection{Testing environment}
As all aforementioned functionality was implemented and ran on a simulated device that does not interact with the host machine operating system in the same way that a typical program would, the testing facilities also had to be a little different. It therefore seems reasonable to shortly discuss how all aforementioned functionality was tested, as any errors in testing usually is quite a bad sign later in the pipeline. The primary file used for testing is the \texttt{kat.py} file in the \texttt{test/} directory.\medskip\\
First of all, to be able to test the scheme the host device and the client device had to be able to communicate. Usually this can be done via some sort of serial port on an actual device, though as this was a virtual device a serial port was not present. The author(s) of the \texttt{pqriscv} project ensured there to be a runtime option that allows for the client device to have a communications port through a linux \texttt{pts} device.\medskip\\
Before any actual communication between client and \texttt{kat.py}, the \texttt{kat.py} script will run some preparations to ensure that it is ready to check for correct outputs from the client device. For \texttt{kat.py} to check correctness it has access to a host-native Rainbow implementation, directly pulled from the github submission repository, which it then compiles and runs before any testing on the client device. This means that before any test, the \texttt{kat.py} file will generate a \texttt{KAT\_<timestamp>} folder that holds the randomseed used, the keys generated from the random seed, the message, an incorrect signature (although this is generated, it is not of importance for the current state of the test-suite) and the signature generated given the keys and message in the folder. The files lying in this folder are then used as the known answers in the known answer tests that \texttt{kat.py} executes.\medskip\\
The initial communication between the test-suite and the client device has the test-suite send all required items to the client device. When key-generation is tested, the input is just the random seed needed for the client device to generate a keypair. For signing and verification the test-script will send the message, a key (public for verification and private for signing) and a signature, being of course only for verification). Once these values have been provided, the script will wait for the client device to answer. For key generation the script checks if the byte-values of the host-native scheme are equal to the byte values of the client scheme. For signing, only signature byte-values are checked, while verification checks for any inconsistencies in the \texttt{true}/\texttt{false} output of the client, still compared to the host-native version.

