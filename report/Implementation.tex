\section{Implementation}
The following section details the differences that came about when porting the reference implementation[Link to github of rainbow] to a platform without an operating system, minimal standard library, and more constrained on memory. It will also detail some of the specifics that the original authors used to implement the functionality of Rainbow for reference in later sections. It will, however, not go into detail on how the optimizations tested differ from the original code, see section \ref{opti} for this.
\subsection{Overview of the reference system}
In general, this port of the Rainbow reference implementation tries to stay as close to its roots as possible, making it easier to argue for correctness and easier to port these findings to other platforms as well (having multiple sources being somewhat similar).\medskip\\
The implementation focuses solely on the standard Rainbow implementation with no key-size adjustments. Focusing on this particular variant of the Rainbow scheme makes the optimizations done clearer for readers new to the Rainbow scheme in general, as it is the closest to how Rainbow is described in many sources on Rainbow and the $MQ$ problem and therefore should be somewhat easier than also having to understand some of the key-size reductions going on in \texttt{CZ-Rainbow} and \texttt{Compressed-Rainbow}.

\subsubsection{Utilities}
Due to the constrained nature of an embedded device and/or FPGA, not all the original functionality provided by the Rainbow authors was needed. In particular, dynamic memory allocation was removed entirely due to the somewhat unstable state of using standard library dynamic memory allocation on embedded devices. Instead, all memory allocations is handled as static memory allocation. Alternatively, one could have implemented a sufficient and secure \texttt{malloc} implementation, though this was not deemed necessary for this project.\medskip\\
The original Rainbow reference implementation also relied on having all data stored in files. As the device running this port of Rainbow has only \texttt{512 kb} of ROM and RAM (each), this was changed. This port of Rainbow relies on a host machine connecting to the embedded device and providing signatures, keys, messages and seeds for (cryptographically secure) randomness.\medskip\\
Although the above two aspects has been altered, they were not aspects that the Rainbow scheme internally was heavily reliant on. 
\subsection{Deep dive into reference functionality}